\documentclass[12pt,a4paper]{article}

% Balíčky pro češtinu
\usepackage[utf8]{inputenc}
\usepackage[czech]{babel}
\usepackage[T1]{fontenc}

% Další užitečné balíčky
\usepackage{amsmath}        % Matematické vzorce
\usepackage{graphicx}       % Obrázky
\usepackage{hyperref}       % Odkazy
\usepackage{geometry}       % Nastavení okrajů
\usepackage{setspace}       % Řádkování

% Nastavení okrajů
\geometry{
    left=2.5cm,
    right=2.5cm,
    top=2.5cm,
    bottom=2.5cm
}

% Metadata dokumentu
\title{Název dokumentu}
\author{Tvoje jméno}
\date{\today}

\begin{document}

\maketitle

\tableofcontents
\newpage

\section{Úvod}

Zde začni psát svůj text. LaTeX automaticky formátuje odstavce.

\subsection{Podsekce}

Můžeš vytvářet podsekce pro lepší strukturu dokumentu.

\section{Hlavní část}

\subsection{Matematické vzorce}

Vzorec v textu: $E = mc^2$

Vzorec na samostatném řádku:
\[
\int_{a}^{b} f(x) \, dx = F(b) - F(a)
\]

\subsection{Seznam}

Odrážkový seznam:
\begin{itemize}
    \item První položka
    \item Druhá položka
    \item Třetí položka
\end{itemize}

Číslovaný seznam:
\begin{enumerate}
    \item První krok
    \item Druhý krok
    \item Třetí krok
\end{enumerate}

\subsection{Tabulka}

\begin{table}[h]
\centering
\begin{tabular}{|l|c|r|}
\hline
Vlevo & Střed & Vpravo \\
\hline
A & B & C \\
D & E & F \\
\hline
\end{tabular}
\caption{Příklad tabulky}
\label{tab:priklad}
\end{table}

\section{Závěr}

Zde napiš závěrečné shrnutí.

% Literatura
\begin{thebibliography}{9}
\bibitem{prvni}
Autor, A. (2024). \textit{Název knihy}. Nakladatelství.

\bibitem{druhy}
Autor, B. (2023). Název článku. \textit{Název časopisu}, 10(2), 123-145.
\end{thebibliography}

\end{document}